% ************************** Thesis Abstract *****************************
% Use `abstract' as an option in the document class to print only the titlepage and the abstract.
\begin{abstract}
This work presents several measurements of the inclusive top pair production cross section at a center of mass energy of $\sqrt{s}=13 \; \mathrm{TeV}$ with the CMS detector.
The cross section is measured using several data sets collected by the CMS detector in 2015 and 2016. The main and most precise result is obtained using full 2016 data set with an integrated
luminosity of $\mathcal{L}_\mathrm{int}=35.9 \;\mathrm{fb}^{-1}$. 

The top quark pair production cross section is measured with a likelihood fit.
The events are required to contain two charged leptons.
The top quark pair production cross section is first measured in the visible phase space,
as defined by the detector acceptance. It is then extrapolated to the full phase space.

The efficiency of the lepton triggers is measured independently. The uncertainty on the trigger efficiency is determined comparing multiple measurement techniques and then propagated
to the measurement of the top quark pair production cross section.

The top quark pole mass is extracted from the cross section measurement, using the next-to-next-to-leading order (NNLO) prediction for the top quark pair production cross section and its measured value. 
\end{abstract}

\chapter*{\centering \Large Zusammenfassung} 

Diese Arbeit beschreibt mehrere Messungen des Wirkungsquerschnittes für Top Quark Paarproduktion mit dem CMS Detektor bei einer Schwerpunktsenergie von $\sqrt{s}=13 \; \mathrm{TeV}$.
Der Wirkungsquerschnitt wird für verschiedene Datensätze gemessen, die in den Jahren 2015 und 2016 von dem CMS Detektor gesammelt wurden.
Das Hauptergebnis wird mit dem vollständigen Datensatz aus 2016 mit einer integrierten Luminosität von $\mathcal{L}_\mathrm{int}=35.9 \;\mathrm{fb}^{-1}$ gemessen.

Der Wirkungsquerschnitt für Top Quark Paarproduktion wird mit einer Maximum-Likelihood-Anpassung für Ereignisse mit zwei geladenen Leptonen gemessen.
Er wird zuerst im sichtbaren Phasenraum gemessen, der durch die Akzeptanz des Detektors definiert wird.
Danach wird der Wirkungsquerschnitt für Top Quark Paarproduktion in den kompletten Phasenraum extrapoliert.

Die Effizienz der Leptontrigger wird separat gemessen. Ihre Unsicherheit wird durch den Vergleich mehrerer Messmethoden bestimmt und dann in der Messung des Wirkungsquerschnitts für Top Quark Paarproduktion verwendet.

Die Polmasse des Top Quarks wird durch den Vergleich des gemessenen Wirkungsquerschnitts für Top Quark Paarproduktion mit einer Vorhersage in nächst-zu-nächst-zu-führender-Ordnung (NNLO) bestimmt.




