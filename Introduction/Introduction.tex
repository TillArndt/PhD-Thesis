%!TEX root = ../thesis.tex
%*******************************************************************************
%****************************** Second Chapter *********************************
%*******************************************************************************

\chapter{Introduction}

Particle physics aims to explore the fundamental structure of the world, the elementary particles and forces.
The Standard Model (SM) is the theory describing particle physics and has successfully predicted experimental results at highest precision. 
It tries to explain the early universe as well as the smallest building blocks of matter. It is able to describe elementary interactions over energy scales of many orders of magnitude. 

One of the important particles in the SM is the top quark. The top quark has a large coupling to the Higgs boson, making it the heaviest known particle.
This high mass leads to a very short lifetime, allowing the study of bare quark properties.
Properties of the top quark, such as the mass, are also important for the determination of Standard Model parameters. Top quarks also have a high production rate, which makes their production
one of the important background processes for measurements of rare SM processes, Higgs boson processes and searches for physics beyond the SM. The top quark is also linked to various models predicting new phenomena beyond the SM.

The top quark pair production cross section is predicted at high precision by SM calculations. Measuring the top quark production cross section with high precision enables a stringent test of these calculations and their underlying assumptions. Confronting the measured top quark pair production cross section with the predicted value allows for the extraction of further Standard Model parameters like the strong coupling, \as, or the mass of the top quark.

With the start of the Large Hadron Collider (LHC) at CERN a new energy frontier opened up in 2009. Until 2012 the LHC provided proton-proton collisions at a center of mass energy of up to $\sqrt{s}=8 \TeV$.  After a two-year break the LHC resumed operations with even higher collision energies, $\sqrt{s}=13 \TeV$, in 2015 through 2018.
The data collected at the new energy frontier offers new opportunities to observe new physics processes and
confirm previous observations.

The Compact Muon Solenoid (CMS) detector is one of the experiments at the LHC. It is a multi-purpose detector.
One of the greatest achievements of the CMS collaboration was the discovery of the Higgs boson, together with the ATLAS collaboration, in 2012~\cite{201230,20121}.   

In the context of this work, three measurements of the top quark pair production cross section were performed.
The first measurement uses data taken in 2015 by the CMS detector with an integrated luminosity of $\lumi= 42 \pbinv$. The first measurement of the top quark pair production
cross section at $\sqrt{s}=13 \TeV$ is designed to test for a possible strong deviation from the SM. The purpose of a second measurement, which uses data taken in 2015 with an integrated luminosity of $\lumi= 2.2 \fbinv$ is to corroborate the result of the first measurement with more data.
The third measurement is the main focus of this work. It uses data taken in 2016 with an integrated luminosity of $\lumi= 35.9 \fbinv$ to measure the top quark pair production cross section with the highest possible precision.

Several steps are taken to ensure a high precision. In order to select a sample with a high number of top quark pair (\ttbar) events, events with two leptons in the final state are chosen for the measurement of the cross section.
Moreover, systematic uncertainties are addressed in detail.
Special emphasis is given to the uncertainty in the determination of the efficiency of the triggers. These triggers select the basic dataset for the cross section measurement and fundamentally determine the reach of this analysis. Therefore, the trigger efficiency and its uncertainty are precisely determined in a separate measurement.
The top quark cross section is then measured with a likelihood fit to kinematic observables, taking into account systematic uncertainties and their correlations. 

Relevant theoretical aspects are described in Chapter~\ref{sec:theo}. This includes a brief summary of top quark physics as well as a description of the simulation.
The experimental setup is described in Chapter~\ref{sec:det}, including the LHC and the CMS detector. The reconstruction of data and simulation is described in Chapter~\ref{sec:SimReco_Reco}. 
The measurement of the trigger efficiency is described in Chapter~\ref{sec:Trigger}. The uncertainty on the trigger efficiency is assessed by comparing multiple methods.
The measurement of the top quark pair production cross section is described in Chapter~\ref{sec:xsec}. The different parts of the measurement such as the event selection and the fit are described in detail.
The agreement of data and simulation is verified for a range of observables. Chapter~\ref{sec:syst_uncert} provides a detailed description of the systematic uncertainties used in the measurement.
The results are given in Chapter~\ref{sec:res}. This chapter also includes the extraction of the top quark pole mass using the measured \ttbar cross section. A study is presented addressing the impact of 
including events with one muon in the final state in the \ttbar cross section measurement. Chapter~\ref{sec:out} summarizes the results and shows
possibilities for future improvement.



