%!TEX root = ../thesis.tex
%*******************************************************************************
%****************************** Second Chapter *********************************
%*******************************************************************************

\chapter{Introduction}

Particle physics aims to explore the fundamental structure of the world, the elementary particles and forces. It tries to explain the early universe as well as the smallest building blocks of matter. It is able to describe interactions over energy scales of many orders of magnitude. 
The Standard Model (SM) is the theory describing particle physics and has successfully predicted experimental results at the highest precision. 

One of the important particles in the SM is the top quark. It has the heaviest known particle, due to its large coupling to the Higgs boson. 
The high mass of the top quark leads to a very short lifetime, allowing the study of bare quark properties.
The properties of the top quark, such as the mass, are also important for the determination of Standard Model parameters. The high production rate for top quarks makes the production of top quarks 
one of the important background processes for a measurements of rare SM processes, Higgs boson processes and searches for physics beyond the SM. Finally, the top quark is also linked to multiple models predicting new phenomena beyond the SM.

The top quark pair production cross section is predicted at high precision by SM calculations. Measuring the top quark production cross section with high precision enables a stringent test of these calculations and their underlying assumptions.Confronting the measured top quark pair production cross section with the predicted value leads to the extraction of further Standard Model parameters like the strong coupling, \as, or the mass of the top quark.

With the start of the Large Hadron Collider (LHC) at CERN a new energy frontier was opened up in 2009. Until 2012 the LHC provided proton-proton collisions at a center of mass energy of up to $\sqrt{s}=8 \TeV$.  After a two year break the LHC resumed operations with even higher collision energies, $\sqrt{s}=13 \TeV$, in 2015.
Data collection at the new energy frontier offers new opportunities to observe new physics processes and
confirm previous observations.

The Compact Muon Solonoid (CMS) detector is one of the experiments at the LHC. It is a multi purpose detector allowing to analyze events containing top quarks.
One of the greatest achievements of the CMS collaboration was the discovery of the Higgs boson, together with the ATLAS collaboration, in 2012 \cite{201230,20121}.   

In this work three measurements of the top quark pair production cross section are presented.
The first measurement uses data taken in 2015 by the CMS detector with an integrated luminosity of $\mathcal{L}_{\mathrm{int}}= 42 \pbinv$. The first measurement of the top quark pair production
cross section at $\sqrt{s}=13 \TeV$ is designed to see a possible strong deviation from the SM. The purpose of second measurement, which uses data with an integrated luminosity of $\mathcal{L}_{\mathrm{int}}= 2.2 \fbinv$ is to corroborate the result of the first measurement with more data.
The third measurement is the main focus of this work. It uses data taken in 2016 with an integrated luminosity of $\mathcal{L}_{\mathrm{int}}= 35.9 \fbinv$ to measure the top quark pair production cross section with the highest possible precision.

In order to select a sample with a high number of top quark pair (\ttbar) events, events with two leptons in the final state are selected for the mesurement of the cross section.
To further achieve a high precision, systematic uncertainties have to be well scrutinized.
Special emphasis is given to the uncertainty on the determination of the efficiency of the triggers. Those triggers select the basic dataset for the cross section measurement and fundamentally determine the reach of the analysis. The trigger efficiency and its uncertainty are precisely determined in a separate measurement.
The top quark cross section is measured with a likelihood fit to kinematic observables, taking into account systematic uncertainties and their correlations. 

Relevant theoretical aspects are described in Chapter \ref{sec:theo}. This includes a brief summary of top quark physics as well as a description of the simulation.
The experimental setup is described in Chapter \ref{sec:det}, including the LHC and the CMS detector. The reconstruction of data and simulation is described in Chapter \ref{sec:SimReco_Reco}. 
The measurement of the trigger efficiency is described in Chapter \ref{sec:Trigger}. The uncertainty on the trigger efficiency is asessed by comparing multiple methods to measure the formér.
The measurement of the top quark pair production cross section is described in Chapter \ref{sec:xsec}. The different parts of the measurement such as the event selection and the fit are described in detail.
The agreement of data and simulation is verified for a range of observables. Chapter \ref{sec:syst_uncert} provides a detailed description of the systematic uncertainties used in the measurement of the \ttbar cross section.
The results are given in Chapter \ref{sec:res}. This chapter also includes the extraction of the top quark pole mass using the measured \ttbar cross section. A study of the impact of including events with one muon in the final state in the \ttbar cross section measurement is presented. Chapter \ref{sec:out} summarizes the results and shows
possibilities for future improvement.



