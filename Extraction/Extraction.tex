%!TEX root = ../thesis.tex
%*******************************************************************************
%****************************** Third Chapter **********************************
%*******************************************************************************
\chapter{The Fitting of the ttbar cross section}

\label{sec:xsec_fit}

In order to measure the cross section several quantities have to be determined in the template fit. Following Equation \ref{eq:CaC} the quantities of interest are the 
acceptance, the efficiency and the number of \ttbar events that is selected. 

The acceptance is defined as the fraction of the visible and the full phase space (see Section \ref{sec:xsec_sel} for the definition of the visible phase space).
Since the measurement can only be done in the visible phase space this value has to be taken from simulation and it can only be constrained within the visible phase space.

The nominal value for the efficiency is taken from simulation as well.  But here, the simulation has been corrected for the efficiency in data as explained in \todo{Link to SFs}. 
As the efficiency for all selected physics objects has been measured in data the efficiency value from simulation corresponds to the efficiency in data within the uncertainties.

In order to measure the \ttbar content in the selected data sample templates taken from simulation are fitted to the data. Under the assumption that these templates are able to sufficiently 
model the signal and background the fit is able to determin amount of \ttbar events in data.
To increase the sensitivity to \ttbar events the events are separated in different categories.

The templates for the fit are separated first separated according to the dilepton decay channel. Since the efficiencies for the muon and electron reconstruction are correlated in all the channels and the \emu channel depends on both this separation allows to constrain the higher of the two lepton related uncertainties.

In order to further increase the separation between signal and background the events are further divided according to the number of b-tagged jets.
There are three categories of events with one, two and zero or more than two b-tagged jets.

This allows an explicit and independent determination of the efficiency of the b-tagging algorithm for data and simulation depending on the nuisance parameters. The efficiency of the b-tagging algorithm should be independent of the rest of the event, so it can be assumed to be the same in all \ttbar decay channels.
This intrinsic measurement of the b-tagging efficiency in the phase space of the measurement is also expected to reduce the impact of the uncertainties on the b-tagging efficiency on the final measurement of the \ttbar cross section.

Following the categorisation of events the variables for the templates themselves are chosen to fullfill two requirements:
They should further increase the separation between signal and background and they should help to reduce uncertainties by being sensitive to the systematic variations.
This choice of templates is described and shown in Section \ref{sec:xsec_templates}.

The details on the fit including the function to be minimized and the parameters involved are given in Section \ref{sec:xsec_stat}.
This includes a discussion of the statistics model and the inclusion of nuisance parameters to account for the systematic uncertainties.

As described above the acceptance is defined as the correction \todo{Better Word ?} from the visible to the full phase space.
Since only the visible phase space is accessible to the measurement the acceptance and its uncertainties are taken from simulation.
In order to be less dependent on the simulation the cross section is first measured in the visible phase space.
It is finally extrapolated to the full phase space as described in Section \ref{sec:xsec_extraction}.


\section{The Choice of Sensitive Templates}
\label{sec:xsec_templates}

The choice of templates is an important part of the fitting procedure.
The variables that are used should both allow to further separate signal and background processes and at the same time be sensitive to systematic
variations in order to constrain their impact on the final result.

As the b-tagging efficiency is already determined intrinsically the fit will already be sensitive to the related nuisance parameters.
Beside the number of b-jets, the number of light jets is one of the main discriminators between \ttbar and background events.
Together with the \pt of the light jets it is also sensitive to the systematic uncertainty on the response of the jet reconstruction and
the systematic uncertainties introduced by theoretical assumptions in the simulation.
These systematic uncertainties make large contributions to the total uncertainty on the \ttbar cross section measurement \todo{Link to CC}, so reducing them is important.

The final templates are shown in Figures \ref{fig:xsec_emu_inputdistr},\ref{fig:xsec_mumu_inputdistr} and \ref{fig:xsec_ee_inputdistr} for the \emu,\mumu and \ee channel respectively. The events are divided first by the dilepton decay channel, then by number of b-tagged jets and 
then by the number of additional light jets resulting in twenty eight distributions overall.

\begin{figure}[htbp!]
  \begin{center}
    \resizebox{0.24 \textwidth}{!}{\includegraphics{CrossSection/Figures/ControlPlots/emu_sysnom/total_0_0_b-jets_step_8.pdf}}
    \resizebox{0.24 \textwidth}{!}{\includegraphics{CrossSection/Figures/ControlPlots/emu_sysnom/total_1_0_b-jets_step_8.pdf}}
    \resizebox{0.24 \textwidth}{!}{\includegraphics{CrossSection/Figures/ControlPlots/emu_sysnom/total_2_0_b-jets_step_8.pdf}}

    \resizebox{0.32 \textwidth}{!}{\includegraphics{CrossSection/Figures/ControlPlots/emu_sysnom/lead_jet_pt_0_1_b-jets_step_8.pdf}}
    \resizebox{0.32 \textwidth}{!}{\includegraphics{CrossSection/Figures/ControlPlots/emu_sysnom/lead_jet_pt_1_1_b-jets_step_8.pdf}}
    \resizebox{0.32 \textwidth}{!}{\includegraphics{CrossSection/Figures/ControlPlots/emu_sysnom/lead_jet_pt_2_1_b-jets_step_8.pdf}}
        
    \resizebox{0.32 \textwidth}{!}{\includegraphics{CrossSection/Figures/ControlPlots/emu_sysnom/second_jet_pt_0_2_b-jets_step_8.pdf}}
    \resizebox{0.32 \textwidth}{!}{\includegraphics{CrossSection/Figures/ControlPlots/emu_sysnom/second_jet_pt_1_2_b-jets_step_8.pdf}}
    \resizebox{0.32 \textwidth}{!}{\includegraphics{CrossSection/Figures/ControlPlots/emu_sysnom/second_jet_pt_2_2_b-jets_step_8.pdf}}

    \resizebox{0.32 \textwidth}{!}{\includegraphics{CrossSection/Figures/ControlPlots/emu_sysnom/third_jet_pt_0_3_b-jets_step_8.pdf}}
    \resizebox{0.32 \textwidth}{!}{\includegraphics{CrossSection/Figures/ControlPlots/emu_sysnom/third_jet_pt_1_3_b-jets_step_8.pdf}}
    \resizebox{0.32 \textwidth}{!}{\includegraphics{CrossSection/Figures/ControlPlots/emu_sysnom/third_jet_pt_2_3_b-jets_step_8.pdf}}  

\caption{Pre-fit distributions (\emu channel) for events with zero as well as three or
  more b-tagged jets (left column): Total event yield for zero (top) and the trailing jet pt for one (second from top),
  two (second from bottom) or three or more (bottom) additional jets. The same for events with one
  b-tagged jet (middle column) and two b-tagged jets (right column) are
  shown below.   
  The hatched bands correspond to the total uncertainty on the predicted number of events. The ratios of the event yields in data and the sum of the
  predicted yields are shown at the bottom of each plot. Here, the solid
  gray band represents the contribution of the statistical uncertainty.  
       \label{fig:xsec_emu_inputdistr}}
  \end{center}
\end{figure}

\begin{figure}[htbp!]
  \begin{center}
    \resizebox{0.240 \textwidth}{!}{\includegraphics{CrossSection/Figures/ControlPlots/mumu_sysnom/total_1_0_b-jets_step_8.pdf}}
    \resizebox{0.240 \textwidth}{!}{\includegraphics{CrossSection/Figures/ControlPlots/mumu_sysnom/total_2_0_b-jets_step_8.pdf}} \\

    \resizebox{0.40 \textwidth}{!}{\includegraphics{CrossSection/Figures/ControlPlots/mumu_sysnom/total_1_1_b-jets_step_8.pdf}}
    \resizebox{0.40 \textwidth}{!}{\includegraphics{CrossSection/Figures/ControlPlots/mumu_sysnom/lead_jet_pt_2_1_b-jets_step_8.pdf}}\\
        
    \resizebox{0.4 \textwidth}{!}{\includegraphics{CrossSection/Figures/ControlPlots/mumu_sysnom/total_1_2_b-jets_step_8.pdf}}
    \resizebox{0.4 \textwidth}{!}{\includegraphics{CrossSection/Figures/ControlPlots/mumu_sysnom/second_jet_pt_2_2_b-jets_step_8.pdf}}\\

    \resizebox{0.4 \textwidth}{!}{\includegraphics{CrossSection/Figures/ControlPlots/mumu_sysnom/total_1_3_b-jets_step_8.pdf}}
    \resizebox{0.4 \textwidth}{!}{\includegraphics{CrossSection/Figures/ControlPlots/mumu_sysnom/third_jet_pt_2_3_b-jets_step_8.pdf}}  
\caption{Pre-fit distributions (\mumu channel): 
  The left column shows events with one b-tagged jet and the total event yield for events with zero (top), one (second from top)
  two (second from bottom) or three or more additional jets (bottom).
  The right column shows events with two b-tagged jets and the total yield for events with zero additional jets (top),
  the trailing jet pt for one (second from top),
  two (second from bottom) or three or more (bottom) additional jets.
  The hatched bands correspond to the total uncertainty on the predicted number of events. The ratios of the event yields in data and the sum of the
  predicted yields are shown at the bottom of each plot. Here, the solid
  gray band represents the contribution of the statistical uncertainty.  
       \label{fig:xsec_mumu_inputdistr}}
  \end{center}
\end{figure}

\begin{figure}[htbp!]
  \begin{center}
    \resizebox{0.24 \textwidth}{!}{\includegraphics{CrossSection/Figures/ControlPlots/ee_sysnom/total_1_0_b-jets_step_8.pdf}}
    \resizebox{0.24 \textwidth}{!}{\includegraphics{CrossSection/Figures/ControlPlots/ee_sysnom/total_2_0_b-jets_step_8.pdf}}\\

    \resizebox{0.4 \textwidth}{!}{\includegraphics{CrossSection/Figures/ControlPlots/ee_sysnom/total_1_1_b-jets_step_8.pdf}}
    \resizebox{0.4 \textwidth}{!}{\includegraphics{CrossSection/Figures/ControlPlots/ee_sysnom/lead_jet_pt_2_1_b-jets_step_8.pdf}}\\
        
    \resizebox{0.4 \textwidth}{!}{\includegraphics{CrossSection/Figures/ControlPlots/ee_sysnom/total_1_2_b-jets_step_8.pdf}}
    \resizebox{0.4 \textwidth}{!}{\includegraphics{CrossSection/Figures/ControlPlots/ee_sysnom/second_jet_pt_2_2_b-jets_step_8.pdf}}\\

    \resizebox{0.4 \textwidth}{!}{\includegraphics{CrossSection/Figures/ControlPlots/ee_sysnom/total_1_3_b-jets_step_8.pdf}}
    \resizebox{0.4 \textwidth}{!}{\includegraphics{CrossSection/Figures/ControlPlots/ee_sysnom/third_jet_pt_2_3_b-jets_step_8.pdf}} 
\caption{Pre-Fit distributions (\ee channel): 
  The left column shows events with one b-tagged jet and the total event yield for events with zero (top), one (second from top)
  two (second from bottom) or three or more additional jets (bottom).
  The right column shows events with two b-tagged jets and the total yield for events with zero additional jets (top),
  the trailing jet pt for one (second from top),
  two (second from bottom) or three or more (bottom) additional jets.
  The hatched bands correspond to the total uncertainty on the predicted number of events. The ratios of the event yields in data and the sum of the
  predicted yields are shown at the bottom of each plot. Here, the solid
  gray band represents the contribution of the statistical uncertainty.  
       \label{fig:xsec_ee_inputdistr}}
  \end{center}
\end{figure}


	
\section{The Fitting Procedure}
\label{sec:xsec_stat}

Following the description in Section \ref{sec:xsec_fit} a binned $\chi^2$ fit is used to extract the cross section and other free parameters.
The basic expression that is minimised can be expressed in the following way:

\begin{equation}
  \chi^2  = \sum_{i} \frac{(n_i-\mu_i)^2}{n_i + \delta_{\mu_i}^2} + \sum_{l} \pi(\omega_l) + \sum_{m} \pi(\lambda_m)
\label{eq:xsec_chisqfunct}
\end{equation}

Here the index $i$ represents a single bin, while $n_i$ is the number of measured events in data, whereas $\mu_i$ is the number
of expected events in simulation. The statistical uncertainty on the expected number of events is introduced in the term $\delta_{\mu_i}$ The terms $\omega_l$ denote the uncertainty on the normalisation of the background contribution and $\lambda_m$
denotes the penalty terms for nuisance parameters with gaussian priors. For these nuisance parameters a unit normal distribution is chosen as penalty term. Nuisance parameters with a uniform prior don't contribute to the penalty terms.
The number of expected events and its uncertainty both depend on the background normalization and all further nuisance parameters.

The events contain a signal as well as a background contribution so $\mu_i$ be written as:

\begin{equation}
\mu_i = s_i(\stt,\vec{\lambda}) 
+ \sum_{l} b_{l,i}(\omega_l,\vec{\lambda}),
\label{eq:xsec_expectev}
\end{equation} 

Here $i$ again denotes the bin, while $l$ denotes the background process. The number of signal events $s_i$ depends on the \ttbar cross section $\stt$ and the nuisance parameters $\vec{\lambda}$.
The number of background events for each background process $b_{l,i}$ also depends on the nuisance parameters and the normalisation of the respective background process $\omega_l$.
In general nuisance parameters related to the detector affect both background and signal, while uncertainties affecting the theory predictions only affect the respective background or signal prediction (see Section \todo{Link to systematics chapter}).


The number of background events as used in Equation \ref{eq:xsec_expectev} can be decomposed as :

\begin{equation}
b_{l,i} = b_{l,i}^{MC} \cdot (1 + \gamma_l \omega_l),
\label{eq:nbli}
\end{equation}

Here $b_{l,i}^{MC}$ denotes the expected number of events from the simulation of the respective background process and $\gamma_l$ denotes its uncertainty.

Following the principles given in Section \ref{sec:xsec_fit}  the number of signal events can be further devided according to the number of b-tagged jets.
Since each of the top quarks decays into a W boson and a b quark it can be assumed that every \ttbar event should contain two jets originating from a b quark.
Any decays of a top quark to a W boson and a light quark can be considered negligible.
It can therefore be assumed that selecting less than two b-tagged jets in a \ttbar event is a measure for the inefficiency of the selection of b-tagged jets.
Following this assumption the number of events in the categories for events with zero or more than two b-tagged jets $s_0$, for events with exactly one $s_1$ and for events with exactly two b-tagged jets $s_2$ can be described as shown in Equation \ref{eq:xsec_nb}. Only events in the \emu channel contribute to category $s_0$.

\begin{eqnarray}
s_0  &=& \mathcal{L}_{\rm int}\stt \epsilon_{ll} \cdot (1-2\epsilon_b(1-C_b\epsilon_b)-C_b\epsilon_b^2) \\
s_1  &=& \mathcal{L}_{\rm int} \stt \epsilon_{ll} \cdot 2 \epsilon_b(1-C_b\epsilon_b) \\
s_2  &=& \mathcal{L}_{\rm int} \stt \epsilon_{ll} \cdot   \epsilon_b^2 C_b 
\label{eq:xsec_nb}.
\end{eqnarray}

Here $\mathcal{L}_{\rm int}$ is the integrated luminosity, $\stt$ is the visible \ttbar cross section and $\epsilon_ll$ is the efficiency of the dilepton selection.
The b-tag efficiency $\epsilon_b$ includes both the efficiency of the kinematic cuts on the b-jet ($\pt > 30\; \GeV, |\eta|<2.4$) and the efficiency of the b-tagging algorithm.
It is generally assumed that the two b-jets can be identified independently of each other. Any remaining correlation is covered by the parameter $C_b$ that can also be written as
$C_b=4s_{ll}s_2/(s_1+2s_2)^2$ where $s_{ll}$ is the total number of selected events. 
The extrapolation to the full phase space using the acceptance will be described in the next section.

Within the fit, the MC simulated quantities $\epsilon_{ll}$, $b_{l,i}$ and $s_{i}$ are taken from simulation and depend on the nuisance parameters $\vec{\lambda}$.
This dependece is modeled with a second order polynomial which is constructed using the nominal and the two systematically varied values of each nuisance parameters $\lambda_m=0,1,-1$.
Some nuisance parameters are based on a one-sided variation so one systematically varied value exists. In these cases the dependence of the simulated quantities is modeled by a linear function.

The MINUIT~\cite{James:1975dr} algorithm is used to minimize the  $\chi^2$ term (see \ref{eq:xsec_chisqfunct} ) as function of the free fit parameters $\stt$, $\vec{\omega}$
and $\vec{\lambda}$. 

\todo{Describe stat model when its final}


\section{Extrapolation from the visible to the Full Phase Space}
\label{sec:xsec_extraction}

The previous explanation only dealt with the visible cross section in the visible phase space. In order to extrapolate that result to the full phase space the acceptance
needs to be considered.

The acceptance can be introduced by replacing the efficiency of the dilepton selection as follows:

\begin{equation}
\epsilon_{ll} = A_{ll} \epsilon^{vis}_{ll}.
\label{eq:epsacc}
\end{equation}

Here $A_{ll}$ is the acceptance and $\epsilon^{vis}_{ll}$ is the efficiency in the visible phase space, with both depending on the nuisance parameters $\vec{\lambda}$.
The acceptance is defined by the kinematic selection requirements on the leptons as given in Section \ref{sec:xsec_sel} applied on the simulation. Specifically the cuts are applied after the parton shower and before the simulation of the detector \todo{Link to simulation description}. There the two leptons are required to be part of the $t \rightarrow W b$ decay. They are further required to be within $|\eta|< 2.4$ with the 
leading lepton having $\pt > 25 \; \GeV$ and the trailing lepton $\pt > 20 \; \GeV$. The invariant mass of the dilepton system is required to be $\mll > 20 \; \GeV$.
These cuts correspond to the selection in the \emu decay channel.

Since the acceptance should only be constrained within the visible phase space, it should be unconstrained for the extrapolation.
This applies to uncertainties on the theory prediction which affect the fraction of events in the visible phase space, especially the variation
of parameters in the matrix element generation and in the parton shower.

Overall the following uncertainties need to be extrapolated: \todo{List of uncerts, with names from syst chamber}

The extrapolation of the uncertainties takes the fitted value of each relevant nuisance parameter as central value. Then the change of acceptance for the explicit $\pm 1 \sigma$ variation is considered as the $\pm 1 \sigma$ variation on the acceptance. These additional uncertainties are then added to the result from the fit in the fiducial phase space for each relevant nuisance parameter. These additional uncertainties are treated as uncorrelated and each is added in quadrature to the result in the fiducial phase space. This procedure can lead to assymetrical variations, even for originally symmetrical nuisance parameter in case the fitted value of the nuisance parameter is not the original central value.





