%!TEX root = ../thesis.tex
%*******************************************************************************
%****************************** Third Chapter **********************************
%*******************************************************************************
\chapter{Measurement of the Top Quark Pair Production Cross Section}

The measurement of the top quark production cross section requires multiple steps.
First, the events that are to be used in the measurement need to be selected as described in Section \ref{sec:xsec_sel}.
Based on this selection the simulation is compared to data to assure that the simulation gives a good estimation of the actual measurement (see Section \ref{sec:xsec_datamc}).
On that dataset the cross section is then measured using a template based binned $\chi^2$ fit as described in Section \ref{sec:xsec_fit}.
The templates are chosen to maximize the precision of the analysis as described in Section \ref{sec:xsec_templates}.
Then the fitting procedure and the statistics model are described in more detail in Section \ref{sec:xsec_stat}.
Finally the extrapolation from the full to the fiducial cross section is discussed in Section \ref{sec:xsec_extraction}.

In general the cross section can be measured according to Equation \ref{eq:CaC}. Here $N_{top}$ denotes the number of selected events containing a top quark pair, $A$ denotes the acceptance of the selection,
$\varepsilon$ denotes the efficiency of the selection and $\mathcal{L}$ denotes the luminosity.

\begin{equation}
\sttbar = \frac{N_{top}}{A \cdot \varepsilon \cdot \mathcal{L}}
\label{eq:CaC}
\end{equation} 

Since the selected sample does not only contain top quark pair events the amount of background needs to be determined.
This is done here by fitting template predictions for all relevant background processes and \ttbar to the data.

\section{Event Selection}
\label{sec:xsec_sel}

The aim of the selection is to select a dataset that is dominated by \ttbar events.
At the same time it also defines acceptance and with it the visual phase space, which should be as broad as possible.
This can partially be solved by using different cuts in the different lepton decay channels and then relying on the principle
of lepton universality in the extrapolation.

All events need to fullfill the trigger selection described in detail in Section \ref{sec:Triggersel}.

In the dilepton channel events need two isolated and well defined leptons \todo{Link to reconstruction chapter}.
Both leptons also need to be within the coverage of the tracker fullfilling the condition of $|\eta| < 2.4$.
For electrons the gap region of the electronic calorimeter $1.4442<|\eta|<1.566$ is vetoed as well.
The decay is then defined according to the two leptons leading in \pt requiring the leading lepton to have $\pt > 25 \GeV$
and the sub leading lepton to have $\pt > 20 \GeV$. 
According to these two leptons the events are then classified into the \emu,\ee and \mumu channels.

The mass of the dilepton system is required to have $\mll > 20 \GeV$ to avoid contamination from low mass DY events.
In the same-flavor channels the window of the Z-mass resonance is vetoed $76 \GeV < \mll <106 \GeV$ to reduce the dominant background of 
resonant Drell-Yan production.

Jets are required to have $\pt > 30 \GeV$ and $|\eta|<2.4$. In order to identify b-tagged jets a tight working point is chosed, giving as high 
purity for selected b-tagged jets.

In the same flavor channels events are required to contain one b-tagged jets, while in the \emu channel does not include any requirement on the jets.

The visible phase space is defined according to the cuts on the \emu channels as it has the highest acceptance:
The leading lepton is required to have $\pt > 25 \GeV$ and the sub leading lepton is required to have $\pt > 20 \GeV$ with the dilepton system
fullfilling $\mll > 20 \GeV$. The principle of lepton universality allows to assume that the regions of the phase space that are additionally cut in the 
same flavor channels can be covered by assuming the same behaviour as in the \emu channel.



\subsection{Comparison of Simulation and Data}
\label{sec:xsec_datamc}

\section{Extracting the Cross Section}
\label{sec:xsec_fit}

The cross section is extracted in a binned $\chi^2$ fit with the \ttbar cross section as free parameter. Templates taken from simulation are fitted to data distributions and systematic uncertainties are treated as nuisance parameters. 
The templates for the fit are separated by dilepton decay channel. For each decay channel they are further sub-divided by the number of b-tagged jets, with categories for one, two and zero or more than two b-tagged jets. This allows an explicit and independent determination of the efficiency of the b-tagging algorithm for data and simulation depending on the nuisance parameter.

\subsection{The Choice of Sensitive Templates}
\label{sec:xsec_templates}
	
\subsection{The Fitting Procedure}
\label{sec:xsec_stat}

\subsection{Extraction from the visual to the Full Phase Space}
\label{sec:xsec_extraction}

