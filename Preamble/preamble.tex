% ******************************************************************************
% ****************************** Custom Margin *********************************

% Add `custommargin' in the document class options to use this section
% Set {innerside margin / outerside margin / topmargin / bottom margin}  and
% other page dimensions
\ifsetCustomMargin
  \RequirePackage[left=37mm,right=30mm,top=35mm,bottom=30mm]{geometry}
  \setFancyHdr % To apply fancy header after geometry package is loaded
\fi

% Add spaces between paragraphs
%\setlength{\parskip}{0.5em}
% Ragged bottom avoids extra whitespaces between paragraphs
\raggedbottom
% To remove the excess top spacing for enumeration, list and description
%\usepackage{enumitem}
%\setlist[enumerate,itemize,description]{topsep=0em}

% *****************************************************************************
% ******************* Fonts (like different typewriter fonts etc.)*************

% Add `customfont' in the document class option to use this section

\ifsetCustomFont
  \usepackage{txfonts}
  % Set your custom font here and use `customfont' in options. Leave empty to
  % load computer modern font (default LaTeX font).
  %\RequirePackage{helvet}

  % For use with XeLaTeX
  %  \setmainfont[
  %    Path              = ./libertine/opentype/,
  %    Extension         = .otf,
  %    UprightFont = LinLibertine_R,
  %    BoldFont = LinLibertine_RZ, % Linux Libertine O Regular Semibold
  %    ItalicFont = LinLibertine_RI,
  %    BoldItalicFont = LinLibertine_RZI, % Linux Libertine O Regular Semibold Italic
  %  ]
  %  {libertine}
  %  % load font from system font
  %  \newfontfamily\libertinesystemfont{Linux Libertine O}
\fi

% *****************************************************************************
% **************************** Custom Packages ********************************

% ************************* Algorithms and Pseudocode **************************

%\usepackage{algpseudocode}


% ********************Captions and Hyperreferencing / URL **********************

% Captions: This makes captions of figures use a boldfaced small font.
%\RequirePackage[small,bf]{caption}

\RequirePackage[labelsep=colon,tableposition=top]{caption}
\renewcommand{\figurename}{Figure} %to support older versions of captions.sty


% *************************** Graphics and figures *****************************

%\usepackage{rotating}
%\usepackage{wrapfig}

% Uncomment the following two lines to force Latex to place the figure.
% Use [H] when including graphics. Note 'H' instead of 'h'
%\usepackage{float}
%\restylefloat{figure}

% Subcaption package is also available in the sty folder you can use that by
% uncommenting the following line
% This is for people stuck with older versions of texlive
%\usepackage{sty/caption/subcaption}
\usepackage{subcaption}

% ********************************** Tables ************************************
\usepackage{booktabs} % For professional looking tables
\usepackage{multirow}

%\usepackage{multicol}
%\usepackage{longtable}
%\usepackage{tabularx}


% *********************************** SI Units *********************************
\usepackage{siunitx} % use this package module for SI units


% ******************************* Line Spacing *********************************

% Choose linespacing as appropriate. Default is one-half line spacing as per the
% University guidelines

% \doublespacing
% \onehalfspacing
% \singlespacing


% ************************ Formatting / Footnote *******************************

% Don't break enumeration (etc.) across pages in an ugly manner (default 10000)
%\clubpenalty=500
%\widowpenalty=500

%\usepackage[perpage]{footmisc} %Range of footnote options


% *****************************************************************************
% *************************** Bibliography  and References ********************

%\usepackage{cleveref} %Referencing without need to explicitly state fig /table

% Add `custombib' in the document class option to use this section
\ifuseCustomBib
   \RequirePackage[square, sort, numbers, authoryear]{natbib} % CustomBib

% If you would like to use biblatex for your reference management, as opposed to the default `natbibpackage` pass the option `custombib` in the document class. Comment out the previous line to make sure you don't load the natbib package. Uncomment the following lines and specify the location of references.bib file

%\RequirePackage[style=numeric-comp, citestyle=numeric, sorting=none, natbib=true]{biblatex}
%\bibliography{References/references} %Location of references.bib only for biblatex

\fi

% changes the default name `Bibliography` -> `References'
\renewcommand{\bibname}{References}


% ******************************************************************************
% ************************* User Defined Commands ******************************
% ******************************************************************************

% *********** To change the name of Table of Contents / LOF and LOT ************

%\renewcommand{\contentsname}{My Table of Contents}
%\renewcommand{\listfigurename}{My List of Figures}
%\renewcommand{\listtablename}{My List of Tables}


% ********************** TOC depth and numbering depth *************************

\setcounter{secnumdepth}{2}
\setcounter{tocdepth}{2}


% ******************************* Nomenclature *********************************

% To change the name of the Nomenclature section, uncomment the following line

%\renewcommand{\nomname}{Symbols}


% ********************************* Appendix ***********************************

% The default value of both \appendixtocname and \appendixpagename is `Appendices'. These names can all be changed via:

%\renewcommand{\appendixtocname}{List of appendices}
%\renewcommand{\appendixname}{Appndx}

% *********************** Configure Draft Mode **********************************

% Uncomment to disable figures in `draft'
%\setkeys{Gin}{draft=true}  % set draft to false to enable figures in `draft'

% These options are active only during the draft mode
% Default text is "Draft"
%\SetDraftText{DRAFT}

% Default Watermark location is top. Location (top/bottom)
%\SetDraftWMPosition{bottom}

% Draft Version - default is v1.0
%\SetDraftVersion{v1.1}

% Draft Text grayscale value (should be between 0-black and 1-white)
% Default value is 0.75
%\SetDraftGrayScale{0.8}


% ******************************** Todo Notes **********************************
%% Uncomment the following lines to have todonotes.

%\ifsetDraft
%	\usepackage[colorinlistoftodos]{todonotes}
%	\newcommand{\mynote}[1]{\todo[author=kks32,size=\small,inline,color=green!40]{#1}}
%\else
%	\newcommand{\mynote}[1]{}
%	\newcommand{\listoftodos}{}
%\fi

% Example todo: \mynote{Hey! I have a note}

% ************Custom definitions
%\usepackage{txfonts}
%\usepackage{fontspec}
%\setmainfont[Path=/usr/share/fonts/truetype/calibri/,
%    BoldItalicFont=CalibriBI.ttf,
%    BoldFont      =CalibriB.ttf,
%    ItalicFont    =CalibriI.ttf]{Calibri.ttf}
\usepackage{xspace}
\usepackage{amsmath}
\usepackage{sty/ptdr-definitions}
\usepackage{longtable}
\usepackage{siunitx}

\newcommand\todo[1]{\textbf{#1}}
\newcommand{\metxy}{\ensuremath{E\!\!\!\!/_\text{x,y}}}
\newcommand{\sttbar}{\ensuremath{\sigma_{\ttbar}}\xspace}
\newcommand{\sttvis}{\ensuremath{\sigma_{\ttbar,\mathrm{vis}}\xspace}}
\newcommand{\mtt}{\ensuremath{m_{\ttbar}}\xspace}
\newcommand{\mtop}{\ensuremath{m_{\mathrm top}}\xspace}
\newcommand{\Wjets}{W+jets\xspace}
\newcommand{\Zjets}{Z+jets\xspace}
\newcommand{\ejets}{e+jets\xspace}
\newcommand{\mujets}{$\mu$+jets\xspace}
\newcommand{\ljets}{$\ell$+jets\xspace}
\newcommand{\mumu}{$\mu^+\mu^-$\xspace}
\newcommand{\ee}{$\mathrm{e^+e^-}$\xspace}
\newcommand{\mue}{$\mu^{\pm}\mathrm{e^{\mp}}$\xspace}
\newcommand{\pb}{\mbox{\ensuremath{\,\text{pb}}}\xspace}
\newcommand{\Pythia} {{\textsc{Pythia}}\xspace} %%%%%%%%%%%%%
\newcommand{\Powheg} {{\textsc{Powheg}}\xspace} %%%%%%%%%%%%%
\newcommand{\Herwig} {{\textsc{Herwig}}\xspace} %%%%%%%%%%%%%
\newcommand{\Herwigpp} {{\textsc{Herwig++}}\xspace} %%%%%%%%%%%%%
\newcommand{\MadSpin} {{\textsc{MadSpin}}\xspace} %%%%%%%%%%%%%
\newcommand{\MGaMCatNLO} {{\textsc{MG5\_aMC@NLO}}\xspace} %%%%%%%%%%%%%
\newcommand{\eepm}{\ensuremath{\Pep\Pem}}
\newcommand{\mmpm}{\ensuremath{\Pgmp \Pgmm}}
\newcommand{\ttpm}{\ensuremath{\Pgt^+ \Pgt^-}}
\newcommand{\empm}{\ensuremath{\Pepm \PGm^\mp}}
\newcommand{\pp}{\ensuremath{\Pp\Pp}}
\newcommand{\ppbar}{\ensuremath{\Pp\Pap}}
\newcommand{\ase}[2]{\ensuremath{_{~- #1}^{~+ #2}}}
\newcommand{\roots}{\ensuremath{\sqrt{s}}}
\newcommand{\lhcE}[1]{\ensuremath{\roots ={#1}~\TeV}}
%\newcommand{\PZ}{\ensuremath{\mathrm{Z}}}
\newcommand{\dy}{\ensuremath{\PZ/\Pgg^\star}}
\newcommand{\dyee}{\ensuremath{\dy\to\eepm}}
\newcommand{\dymm}{\ensuremath{\dy\to\mmpm}}
\newcommand{\dytt}{\ensuremath{\dy\to\ttpm}}
\newcommand{\mll}{\ensuremath{M_{\ell\ell}}\xspace}
\def\mrm{\mathrm}
\newcommand{\isocomb}{\ensuremath{I_\mrm{comb}}}
\newcommand{\WoZ}{\ensuremath{\PW/\PZ}}
\providecommand{\POWHEG} {\textsc{Powheg}\xspace}
\providecommand{\PYTHIA} {\textsc{Pythia}\xspace}
\providecommand{\HERWIGPP} {\textsc{Herwig++}\xspace}
\newcommand{\tW}{\ensuremath{\mathrm{t}\PW}}
\newcommand{\VV}{\ensuremath{\mathrm{VV}}}
\newcommand{\ns}{\ensuremath{\mrm{ns}}}
\newcommand{\wmn}{\ensuremath{\PW\to\Pgm\Pgngm}}
\newcommand{\met} {\ensuremath{E\!\!\!\!/_T}}
\renewcommand{\MET}{\mbox{$\not \!\! E_T$}}
\newcommand{\pythia}{{\sc{Pythia}}}
\renewcommand{\ttbar}{\ensuremath{\mathrm{t}\bar{\mathrm{t}}}\xspace}
\newcommand{\invpb}{pb$^{-1}$}
\newcommand{\geVcc}{GeV}
\providecommand{\ee}{\ensuremath{ee}\xspace}
\newcommand{\emu}{\ensuremath{\mathrm{e}^\pm\mu^\mp}\xspace}
\newcommand{\mev}{\ensuremath{\mathrm{\;MeV}}\xspace}
\newcommand{\tev}{\ensuremath{\mathrm{\;TeV}}\xspace}
\newcommand{\mevc}{\ensuremath{\mathrm{\;MeV}}\xspace}
\newcommand{\gevc}{\ensuremath{\mathrm{\;GeV}}\xspace}
\newcommand{\kevcc}{\ensuremath{\mathrm{\;keV}}\xspace}
\newcommand{\mevcc}{\ensuremath{\mathrm{\;MeV}}\xspace}
\newcommand{\gevcc}{\ensuremath{\mathrm{\;GeV}}\xspace}
%%% analysis results 
\newcommand{\resultxsecmain}{\ensuremath{827 \pm  2 ({\rm stat}) \pm 24 ({\rm syst}) \pm 21 ({\rm lumi}) \pb}\xspace}
\newcommand{\resultxsecvismain}{\ensuremath{24.88 \pm 0.05({\rm stat}) \pm 0.65 ({\rm syst}) \pm 0.62({\rm lumi})\pb}\xspace}
\newcommand{\uncertaintytotmain}{\ensuremath{32 \pb~(3.83\%)}\xspace}
\newcommand{\xsectheo}{\ensuremath{ 832 \pm^{20}_{29} {\rm (scale)} \pm 35({\rm PDF}+\alpha_s) \pb}\xspace}

\renewcommand{\lumi}{\mathcal{L}_\mathrm{int}}
\newcommand{\lumiv}{35.9 fb$^{-1}$}
\newcommand{\lumivwunc}{35.9 $\pm$ 0.9 fb$^{-1}$}

\newcommand{\as}{\ensuremath{\alpha_\mathrm{S}}\xspace}
\newcommand{\asq}{\ensuremath{\alpha_\mathrm{S}(Q)}\xspace}
\newcommand{\asmz}{\ensuremath{\alpha_\mathrm{S}(m_Z)}\xspace}
\newcommand{\mur}{\ensuremath{\mu_\mathrm{R}}\xspace}
\newcommand{\muf}{\ensuremath{\mu_\mathrm{F}}\xspace}
\newcommand{\stt}{\ensuremath{\sigma_\mathrm{t\bar{t}}}\xspace}
\newcommand{\msbar}{\ensuremath{\mathrm{\overline{MS}}}\xspace}
\newcommand{\mtmt}{\ensuremath{m_\mathrm{t}(m_\mathrm{t})}\xspace}
\newcommand{\mtp}{\ensuremath{m_\mathrm{t}^{\mathrm{pole}}}\xspace}
\newcommand{\mtMC}{\ensuremath{m_\mathrm{t}^{\mathrm{MC}}}\xspace}

 \renewcommand{\maketitle}{
 \begin{titlepage}

   \thispagestyle{empty}
   \begin{center}
     \null
     {\huge \textbf{
      Precision measurement of the \\
     top quark pair production cross section 
      at $\boldmath{\sqrt{s}=13 \; \mathrm{TeV}}$ with the CMS detector}\par}

    
     \vspace{2.1cm}
    
     {\Large \bf Dissertation\\}
    
     \vspace{0.2cm}
     {\large
       zur Erlangung des Doktorgrades\\
       an der Fakult\"{a}t f\"{u}r Mathematik, Informatik und Naturwissenschaften\\
       Fachbereich Physik\\
       der Universit\"{a}t Hamburg\\
       \vspace{2.0cm}
       vorgelegt von\\
       \vspace{0.5cm}
       {\Large \textsc{Till Arndt}\\
         \normalsize aus Frankfurt am Main}
       \vspace{0.2cm}
      
       \vspace{2.6cm}
      
       Hamburg\\
       2018\\
      } 
   \end{center}
      \newpage
      \thispagestyle{empty}
      \null
      \vfill
      \hspace{-0.5cm}
      \begin{tabular}{ll}
        Gutachter der Dissertation: & PD. Dr. Andreas Meyer\\
                                       & Prof. Dr. Johannes Haller\\[3mm]
        Zusammensetzung der Pr\"{u}fungskommission: 
        & Prof. Dr. Caren Hagner\\
        & Prof. Dr. Gudrid Moortgat-Pick \\
        & Prof. Dr. Johannes Haller\\
        & PD. Dr. Andreas Meyer\\
        & Dr. Roberval Walsh\\[3mm]
        Vorsitzender der Pr\"{u}fungskommission: & Prof. Dr. Caren Hagner\\[3mm]
        Datum der Disputation: & 5.07.2018\\[3mm]
        Vorsitzender Fach-Promotionsausschusses PHYSIK: & Prof. Dr. Wolfgang Hansen\\[3mm]
        Leiter des Fachbereichs PHYSIK: & Prof. Dr. Michael Potthoff\\[3mm]
        Dekan der Fakult\"{a}t MIN: & Prof. Dr. Heinrich Graener\\     
      \end{tabular}

 \end{titlepage}
 }