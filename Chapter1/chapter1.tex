%!TEX root = ../thesis.tex
%*******************************************************************************
%*********************************** First Chapter *****************************
%*******************************************************************************

\chapter{Trigger Efficiency measurement}  %Title of the First Chapter

The uncertainty on the trigger efficiency determination has been one of the dominant systematic uncertainties in \sttbar measurements.
Consequently, measuring the trigger efficiency with the highest possible precision is an important part of improving the precision of these
measurements.

This chapter describes the trigger selection used in this measurement and how this selection can already help to improve
the later efficiency measurement (see Section \ref{sec:TriggerSel}). It is further explained how the efficiency of this trigger selection is measured and
show the results for this measurement (see Section \ref{sec:TriggerMetMethod}). In order to estimate the precision of this measurement the trigger efficiency
is also measured with another method (see Section \ref{sec:TriggerTPMethod}). The two results are finally compared and the chapter concludes with defining the uncertainty
of the trigger efficiency measurement (see Section \ref{sec:TriggerComp}). 



%********************************** %First Section  **************************************
\section{Trigger Selection} %Section - 1.1 
\label{sec:TriggerSel}

The trigger selection is optimized for maximum efficiency. Because of the statistical properties of the binomial distribution a higher efficiency
leads to a lower uncertainty. The efficiency itself is measured with respect to the final reconstruced (alsso called offline) objects (see Section XX for details
on this reconstruction).

For a di-lepton selection the starting point are of course the di-lepton triggers. There are different triggers depending on the flavor of each of the
two leptons. These triggers typically have assymetric cuts on the \pt of the two leptons, so the leading lepton has a significantly higher pt requirement than
the trailing lepton. In general the quantities reconstructed for the trigger are slightly different from the objects used in the offline analysis. This
leads to an in-efficiency of the trigger compared to the offline selection close to the kinematic requirements for the trigger objects.

In the case of the di-lepton trigger this mainly effects the efficiency at the two \pt cut-offs for the leptons. One way to mitigate this is to
require a higher offline \pt for the leptons than is required by the trigger \todo{Ref to selection}.
Another way to reduce this inefficiency is to combine the di-lepton trigger with other triggers which only depend on one lepton, mitigating any
influence of the second lepton. This leads to a combination of di-lepton and single lepton triggers with a logical "or".
In general only unprescaled triggers are used in this analysis.

This combination increases the trigger efficiency by $\sim 10 \; \%$ compared to onyl using the di-lepton trigger.
This is shown in Figure \ref{fig:TriggerSel} for a dataset taken in 2016 in Run B to F.

\begin{figure}[htbp!]
  \begin{center}
    \resizebox{0.48 \textwidth}{!}{\includegraphics{Chapter1/Figures/Trigger_pt_nosilep.png}}
    \resizebox{0.48 \textwidth}{!}{\includegraphics{Chapter1/Figures/Trigger_eta_nosilep.png}}
    \resizebox{0.48 \textwidth}{!}{\includegraphics{Chapter1/Figures/Trigger_pt.png}}
    \resizebox{0.48 \textwidth}{!}{\includegraphics{Chapter1/Figures/Trigger_eta.png}}
      \caption{Efficiencies of the trigger selection in the \emu channel for simulation and data and the corresponding scale factor. The top row shows the efficiency when using dilepton triggers only, while the bottom row shows
      the efficiency for a combination of dilepton and single lepton triggers. The left row shows the efficiency depending on the \pt of the leading lepton. The right column shows the efficiency depending on the $\eta$ of the leading lepton. }  
       \label{fig:TriggerSel}
  \end{center}
\end{figure}


The final trigger selection is a bit more complicated because we aim to take unprescaled triggers with the most relaxed requirements possible.
This results in the trigger selection shown in Table \ref{tab:triggerSel} for each of the three di-lepton channels.
Double counting events is prevented by explicitly requiring triggers on their respective dataset, but vetoing them on the other datasets.

 \begin{table}[hbt]
    \centering
    \caption{The trigger selection that is applied on data. It can differ between the different run periods because of changing prescales.}
    \label{tab:triggerSel}
     \begin{tabular}
            {c|c|c}
            channel & run & trigger \\
            \hline
             \mumu & B-G & HLT\_Mu17\_TrkIsoVVL\_Mu8\_TrkIsoVVL \\
              & B-G & HLT\_Mu17\_TrkIsoVVL\_TkMu8\_TrkIsoVVL \\
              & H & HLT\_Mu17\_TrkIsoVVL\_Mu8\_TrkIsoVVL\_DZ \\
              & H & HLT\_Mu17\_TrkIsoVVL\_TkMu8\_TrkIsoVVL\_DZ \\
              & B-H & HLT\_IsoMu24 \\
              & B-H & HLT\_IsoTkMu24 \\
            \hline
             \ee &  B-H & HLT\_Ele23\_Ele12\_CaloIdL\_TrackIdL\_IsoVL\_DZ \\
              &  B-H & HLT\_Ele27\_WPTight\_Gsf \\
            \hline
             \empm & B-G & HLT\_Mu23\_TrkIsoVVL\_Ele12\_CaloIdL\_TrackIdL\_IsoVL \\
              & B-G & HLT\_Mu8\_TrkIsoVVL\_Ele23\_CaloIdL\_TrackIdL\_IsoVL \\
              & H & HLT\_Mu23\_TrkIsoVVL\_Ele12\_CaloIdL\_TrackIdL\_IsoVL\_DZ \\
              & H & HLT\_Mu8\_TrkIsoVVL\_Ele23\_CaloIdL\_TrackIdL\_IsoVL\_DZ \\
              & B-H & HLT\_Ele27\_WPTight\_Gsf \\
              & B-H & HLT\_IsoMu24 \\
              & B-H & HLT\_IsoTkMu24 \\
    \end{tabular}
\end{table}







%********************************** %Second Section  *************************************
\section{Method: Independent Triggers} %Section - 1.2
\label{sec:TriggerMetMethod}

In this analysis the trigger efficiency is mainly measured using a dataset taken with an orthogonal trigger selection.
In general this allows to measure the trigger efficiency without bias with the same method in data and simulation independently 
of each other. It is important that these independent triggers are indeed as independent from the dilepton triggers as possible
and for simplicity they should also be unprescaled in the full run range.

Here triggers based on missing transverse energy (\ETm) are used. They are especially usefull for a \ttbar analysis, 
as dileptonic \ttbar events should produce \ETm due to the two neutrinos. This allows to measure the trigger efficiency for data
in an event sample which at least contains \ttbar events. The efficiency in simulation can be measured on a sample of \ttbar events.

The trigger efficiency is then measured according to the formula given in Equation \ref{eq:TriggerEff}, where $N$ denotes the number of 
events fullfilling the given requirements.

\begin{equation}
\varepsilon_{trig} = \frac{N(\mathrm{Offline Selection + \ETm \; Trigger + Measured Trigger})}{N(\mathrm{Offline Selection + \ETm \; Trigger})}
\label{eq:TriggerEff}
\end{equation}

The efficiency is measured for each event, consequently the resulting scale factors to be applied on simulation to correct to data are applied per event as well.
This is especially useful if a combination of multiple triggers is used in the selection.

The statistical uncertainty is calculated assuming a binomial distribution using the Clopper-Pearson method \todo{cit PDG}. It offers at least nominal coverage 
of the given confidence level and preesents the conservative option to estimate these uncertainties. 

Results for efficiencies and scale factors for the \ee, \mumu and \emu channel are shown in figures \ref{fig:MET_ee},\ref{fig:MET_mumu} and \ref{fig:MET_emu} respectively.

\begin{figure}[htbp!]
  \begin{center}
    \resizebox{0.48 \textwidth}{!}{\includegraphics{Chapter1/Figures/MET/ee/leading_eta}}
    \resizebox{0.48 \textwidth}{!}{\includegraphics{Chapter1/Figures/MET/ee/seleading_eta}}
    \resizebox{0.48 \textwidth}{!}{\includegraphics{Chapter1/Figures/MET/ee/leading_pt}}
    \resizebox{0.48 \textwidth}{!}{\includegraphics{Chapter1/Figures/MET/ee/seleading_pt}}
    \resizebox{0.48 \textwidth}{!}{\includegraphics{Chapter1/Figures/MET/ee/jet_multi}}
    \resizebox{0.48 \textwidth}{!}{\includegraphics{Chapter1/Figures/MET/ee/vertex_multi}}  
      \caption{Efficiencies of the trigger selection in the \ee channel for simulation and data and the corresponding scale factor. The top row shows the efficiency depending on $\eta$ of the leading (left) and trailing (right) lepton. The middle row shows the effciency \pt of the leading (left) and trailing (right) lepton. The bottom rwo shows the efficiency depending on the jet multiplicity on the left and the vertex multiplicity on the right.
       The error bars denote statistical uncertainties. }  
      
    \label{fig:MET_ee}
  \end{center}
\end{figure}

\begin{figure}[htbp!]
  \begin{center}
    \resizebox{0.48 \textwidth}{!}{\includegraphics{Chapter1/Figures/MET/mumu/leading_eta}}
    \resizebox{0.48 \textwidth}{!}{\includegraphics{Chapter1/Figures/MET/mumu/seleading_eta}}
    \resizebox{0.48 \textwidth}{!}{\includegraphics{Chapter1/Figures/MET/mumu/leading_pt}}
    \resizebox{0.48 \textwidth}{!}{\includegraphics{Chapter1/Figures/MET/mumu/seleading_pt}}
    \resizebox{0.48 \textwidth}{!}{\includegraphics{Chapter1/Figures/MET/mumu/jet_multi}}
    \resizebox{0.48 \textwidth}{!}{\includegraphics{Chapter1/Figures/MET/mumu/vertex_multi}}  
      \caption{Efficiencies of the trigger selection in the \mumu channel for simulation and data and the corresponding scale factor. The top row shows the efficiency depending on $\eta$ of the leading (left) and trailing (right) lepton. The middle row shows the effciency \pt of the leading (left) and trailing (right) lepton. The bottom rwo shows the efficiency depending on the jet multiplicity on the left and the vertex multiplicity on the right.
      The error bars denote statistical uncertainties. }  
      
    \label{fig:MET_mumu}
  \end{center}
\end{figure}

\begin{figure}[htbp!]
  \begin{center}
    \resizebox{0.48 \textwidth}{!}{\includegraphics{Chapter1/Figures/MET/emu/leading_eta}}
    \resizebox{0.48 \textwidth}{!}{\includegraphics{Chapter1/Figures/MET/emu/seleading_eta}}
    \resizebox{0.48 \textwidth}{!}{\includegraphics{Chapter1/Figures/MET/emu/leading_pt}}
    \resizebox{0.48 \textwidth}{!}{\includegraphics{Chapter1/Figures/MET/emu/seleading_pt}}
    \resizebox{0.48 \textwidth}{!}{\includegraphics{Chapter1/Figures/MET/emu/jet_multi}}
    \resizebox{0.48 \textwidth}{!}{\includegraphics{Chapter1/Figures/MET/emu/vertex_multi}}  
      \caption{Efficiencies of the trigger selection in the \emu channel for simulation and data and the corresponding scale factor. The top row shows the efficiency depending on $\eta$ of the leading (left) and trailing (right) lepton. The middle row shows the effciency \pt of the leading (left) and trailing (right) lepton. The bottom rwo shows the efficiency depending on the jet multiplicity on the left and the vertex multiplicity on the right.
      The error bars denote statistical uncertainties. }  
      
    \label{fig:MET_emu}
  \end{center}
\end{figure}


%********************************** % Third Section  *************************************
\section{Method: Tag and Probe}  %Section - 1.3 
\label{sec:TriggerTPMethod}

In order to provide an alternative and totally independent trigger efficiency measurement the Tag and Probe method is used.
Here the trigger efficiency can be measured independently in data and simulation by using the decay of the Z-boson into two leptons.

In general one of the leptons (the Tag) is required to pass a tight selection. This selection includes matching the offline lepton to the lepton reconstructed at trigger level by requiring the two to have a maximum distance and difference in \pt. The second lepton (the Probe) is only required to pass a very loose selection, so it should be as unbiased as possible. The invariant mass of the two leptons then needs to be within the Z-mass window, in which case it can generally be assumed that the second lepton is a real lepton and not a fake.
In order to measure the efficiency of a certain requirement the probe can then either fullfill it ("passing probe") or fail it ("failing probe"). The efficiency is then determined by dividing the number of passing probes by all probes.

A residual fake contamination is determined by fitting both background and signal distributions for passing and failing probes. The background distribution is often described by an exponential function. The signal has a more complicated functional form, it can for example be parametrized by a double voigtian function. When the efficiency is measured on data it is also possible to fit templates from MC.
In contrast for the efficiency determination in simulation a fit is sometimes not needed as the fake contamination can be negligible.

The efficiency can also be measured differential depending on kinematic properties of the lepton. It commonly depends on both the $\eta$ and the \pt of the lepton. 

In contrast to the efficiency measurement using orthogonal triggers (see Section \ref{sec:TriggerMetMethod}) the Tag and Probe measurement depends on the lepton itself. It measures the efficiency for each lepton, consequently several measurements have to be combined for a multi lepton trigger or an efficiency measurement with multiple triggers (see Section \ref{sec:TriggerComp}). 

Examples for the efficiency measurement in one bin as well as the overall scale factor for the muonic part ("muon leg") of the\\ HLT\_Mu23\_TrkIsoVVL\_Ele12\_CaloIdL\_TrackIdL\_IsoVL trigger both for data and simulation and the resulting overall scale factors are shown in Figure \ref{fig:TP_Mu23}.

\begin{figure}[htbp!]
  \begin{center}
    \resizebox{0.48 \textwidth}{!}{\includegraphics{Chapter1/Figures/TaP/Mu23_data_fits}}
    \resizebox{0.48 \textwidth}{!}{\includegraphics{Chapter1/Figures/TaP/Mu23_MC_fits}}\\
    \resizebox{0.48 \textwidth}{!}{\includegraphics{Chapter1/Figures/TaP/Mu23_sf}}
      \caption{Results of the trigger efficiency measurement with the Tag and Probe method for the Mu23 trigger leg. The upper row shows the measurement for one bin for each data (left) and simulation (right). The lower row shows 
       the scale factor between data and simulation binned in $\eta$ and \pt. }  
    \label{fig:TP_Mu23}
  \end{center}
\end{figure}

For the efficiency measurement the electron part ("electron leg") of the\\ HLT\_Mu8\_TrkIsoVVL\_Ele23\_CaloIdL\_TrackIdL\_IsoVL trigger for data and simulation and the corresponding scale factors sre shown in Figure \ref{fig:TP_Ele23}.
Here the efficiency on simulation is not determined with a fit, but by counting the events.

\begin{figure}[htbp!]
  \begin{center}
    \resizebox{0.48 \textwidth}{!}{\includegraphics{Chapter1/Figures/TaP/Ele23_data_fit}}
    \resizebox{0.48 \textwidth}{!}{\includegraphics{Chapter1/Figures/TaP/Ele23_data_eff}}
      \caption{Results of the trigger efficiency measurement with the Tag and Probe method for the Ele23 trigger leg. In the left plot the fit for one example bin in data is shown. The right plot shows the efficiency in data and simulation in bins of $\eta$ with multiple colored graphs showing the \pt dependence. Below that the scale factors are shown in the same format. }  
    \label{fig:TP_Ele23}
  \end{center}
\end{figure}


%********************************************** Fourth Section


\section{Comparison of Trigger Efficiency Measurements}
\label{sec:TriggerComp}

The two trigger efficiency measurements described in Sections \ref{sec:TriggerMetMethod} and \ref{sec:TriggerMetMethod} have one fundamental difference: The orthogonal trigger method measures the trigger efficiency for a complete trigger selection (multiple triggers) and per each event, which is mainly defined by the existence of two isolated leptons fullfilling some kinematic cuts. In contrast, the Tag and Probe method measures the trigger efficiency for each lepton of each trigger separately. These per leg efficiencies need to be combined to the efficiency for each di-lepton trigger. This is done by multiplying the efficiencies of the single trigger legs. 
Then the efficiencies of each of the lepton triggers involved in the trigger selection needs to be combined. Since they are combined with a logical OR, this is done by multiplying the in-efficiencies of the trigger as in Equation \ref{eq:TPcombine}.

\begin{equation}
\varepsilon(trig_{combined}) = 1- (1-\varepsilon(trig_A))\cdot(1- \varepsilon(trig_B))
\label{eq:TPcombine}
\end{equation}

The combined trigger efficiency measured with the Tag and Probe method can be applied per event for the complete trigger selection. It can be then be compared to the trigger efficiency measured with the independent triggers.
For this comparison both efficiencies are applied to the same simulated \ttbar sample. These \ttbar events are required to contain an electron and a muon \textbf{TODO: reference to selection} and the trigger efficiency scale factors are applied depending on $\eta$ of these leptons.
After the rescaling, properties of these two sets of \ttbar events are compared as shown in Figure \ref{fig:Clos_emu}.

The distributions are consistent with each other in nearly the whole range of phase space. The largest disagreement is found for events with electrons with $\pt<20\GeV$, but these events are not considered for the final trigger efficiency scale factors. This agreement shows that there is no systematic difference between the two methods to measure a trigger efficiency in the \ttbar phase space. This does not necessarily apply to any other possible phase space, but the trigger efficiency measurement here specifically applies to the \ttbar phase space. This agreement is considered for the determination of the systematic uncertainties on the trigger efficiency.

\begin{figure}[htbp!]
  \begin{center}
    \resizebox{0.48 \textwidth}{!}{\includegraphics{Chapter1/Figures/Clos/emu_ptele}}
    \resizebox{0.48 \textwidth}{!}{\includegraphics{Chapter1/Figures/Clos/emu_etaele}}\\
    \resizebox{0.48 \textwidth}{!}{\includegraphics{Chapter1/Figures/Clos/emu_ptmu}}
    \resizebox{0.48 \textwidth}{!}{\includegraphics{Chapter1/Figures/Clos/emu_etamu}} \\
    \resizebox{0.48 \textwidth}{!}{\includegraphics{Chapter1/Figures/Clos/emu_mll}}
    \resizebox{0.48 \textwidth}{!}{\includegraphics{Chapter1/Figures/Clos/emu_njets}}        
      \caption{Comparison of \ttbar simulation reweighted according to the trigger efficiencies measured with the Tag and Probe method and with independent MET triggers.
       The top row shows the comparison in bins of \pt (left) and $\eta$ (right) of the electron. The middle row shows it in bins of \pt (left) and $\eta$ (right) of the muon. The bottom row shows it in bins of the invariant mass of the dilepton system(left) and the number of jets (right). The uncertainties shown are statistical. The lower panels show the ratio of the distribution reweighted with the efficiency from the MET trigger method divided by the distribution reweighted by the Tag and Probe method. }  
    \label{fig:Clos_emu}
  \end{center}
\end{figure}



